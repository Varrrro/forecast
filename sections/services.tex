\section{Construcción y despliegue de los servicios}
Los ficheros con los modelos entrenados son usados por los servicios
de predicción en tiempo de ejecución. Estos servicios han sido implementados
en Python con la librería \textbf{\textit{hug}}\footnote{\href{https://www.hug.rest/}{Web de \textit{hug}}}
y su despliegue se realiza mediante contenedores Docker.

\subsection{Funcionamiento}
Cada servicio consta de dos partes: el punto o ruta de acceso y el modelo
predictivo. Para el punto de acceso o \textit{endpoint} se define un
controlador con \textit{hug}, como ya hemos indicado. Este controlador
obtiene el número de horas al que se quiere realizar la predicción y,
con él, llama al modelo para que realice la predicción.

Al iniciar el servicio, los modelos se cargan a partir de los ficheros dados.
Cuando realiza la predicción, los resultados devueltos por los modelos
de temperatura y humedad se combinan en una única lista que se devuelve en
formato JSON.

Como se trata de dos servicios separados y que se ejecutan en contenedores
independientes, se tiene que implementar un \textit{gateway} que sirva como
punto de acceso unificado para ambos. Sencillamente, es un \textit{proxy}
inverso que redirige las peticiones recibidas a cada servicio dependiendo
de la ruta indicada. Este \textit{gateway} se ha implementado en Go.

\begin{itemize}
    \item\href{
        https://github.com/Varrrro/forecast/blob/master/src/v1/arima_rest.py
    }{API de la versión 1 del servicio}
    \item\href{
        https://github.com/Varrrro/forecast/blob/master/src/v2/autoreg_rest.py
    }{API de la versión 2 del servicio}
    \item\href{
        https://github.com/Varrrro/forecast/blob/master/src/v1/arima.py
    }{Modelo de la versión 1 del servicio}
    \item\href{
        https://github.com/Varrrro/forecast/blob/master/src/v2/autoreg.py
    }{Modelo de la versión 2 del servicio}
    \item\href{
        https://github.com/Varrrro/forecast/blob/master/src/gateway/gateway.go
    }{\textit{Gateway} del sistema}
\end{itemize}

\subsection{Descarga del código fuente}
Todo este código se encuentra alojado en GitHub, por lo que para incorporarlo
en el flujo de trabajo tan solo tenemos que clonar el repositorio mediante un
\textit{BashOperator}. El repositorio se clona al directorio \lstinline{/tmp/forecast/code}.

\begin{itemize}
    \item\href{
        https://github.com/Varrrro/forecast/blob/master/airflow/tasks.py#L147-L152
    }{Clonación del repositorio de GitHub}
\end{itemize}

\subsection{Ejecución de pruebas}
Se han implementado con \textit{unittest} algunas pruebas unitarias sencillas
para la función de los modelos que se encarga de realizar la predicción y
unir los resultados de temperatura y humedad. Antes de desplegar los servicios,
se ejecutan estas pruebas para comprobar que su implementación es correcta.

\begin{itemize}
    \item\href{
        https://github.com/Varrrro/forecast/blob/master/tests/test_arima.py
    }{Pruebas del modelo ARIMA}
    \item\href{
        https://github.com/Varrrro/forecast/blob/master/tests/test_autoreg.py
    }{Pruebas del modelo autorregresivo}
    \item\href{
        https://github.com/Varrrro/forecast/blob/master/airflow/tasks.py#L154-L159
    }{Ejecución de las pruebas}
\end{itemize}

\subsection{Construcción de imágenes}
Tanto el \textit{gateway} como las dos versiones de los servicios tienen su
propio fichero \lstinline{Dockerfile} que define la construcción de las
imágenes de los contenedores que se lanzarán.

Todas estas imágenes exponen el puerto 8080, adonde deben llegar las peticiones.
Además, las imágenes de las dos versiones de la API definen un punto para
montar un volumen que es el que debe contener los ficheros de los modelos.

En el caso de la imagen del \textit{gateway}, aprovechamos que Go es un
lenguaje compilado para aplicar una construcción multifase, lo que nos permite
tener una imagen final de tamaño mínimo.

Las imágenes se construyen usando \lstinline{docker build} con \textit{BashOperator}.

\begin{itemize}
    \item\href{
        https://github.com/Varrrro/forecast/blob/master/src/v1/Dockerfile
    }{Dockerfile de la versión 1 de la API}
    \item\href{
        https://github.com/Varrrro/forecast/blob/master/src/v2/Dockerfile
    }{Dockerfile de la versión 2 de la API}
    \item\href{
        https://github.com/Varrrro/forecast/blob/master/src/gateway/Dockerfile
    }{Dockerfile del \textit{gateway}}
\end{itemize}

\subsection{Creación de red Docker}
Cada una de las partes del sistema (servicios y \textit{gateway}) se ejecuta
en un contenedor independiente. Es por ello que necesitamos crear una red
Docker que permita al contenedor del \textit{gateway} comunicarse con los
de los servicios.

\begin{itemize}
    \item\href{
        https://github.com/Varrrro/forecast/blob/master/airflow/tasks.py#L182-L187
    }{Creación de la red Docker}
\end{itemize}

\subsection{Lanzamiento de los contenedores}
Con las imágenes construidas y la red Docker creada, ya podemos lanzar los
contenedores del sistema. Para ello, usamos \lstinline{docker run}, teniendo
en cuenta que todos los contenedores deben conectarse a la red (Con
\lstinline{--network}) y que el contenedor del \textit{gateway} debe enlazar
su puerto 8080 con un puerto local (con \lstinline{-p}).

El orden en el que se despliegan estos contenedores también es importante,
ya que el \textit{gateway} depende de los servicios, por lo que éstos deben
desplegarse primero.

\begin{itemize}
    \item\href{
        https://github.com/Varrrro/forecast/blob/master/airflow/tasks.py#L189-L208
    }{Ejecución de los contenedores}
\end{itemize}
